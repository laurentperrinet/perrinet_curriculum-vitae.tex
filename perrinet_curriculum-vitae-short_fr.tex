%!TEX TS-program = lualatex
%!TEX encoding = UTF-8 Unicode
%------------------------------------
% Dario Taraborelli
% Mise en page de votre CV académique en LaTeX
%
% URL : http://nitens.org/taraborelli/cvtex
% AVERTISSEMENT : Ce modèle est fourni gratuitement et sans garantie qu'il compilera correctement sur votre système si vous avez une configuration non standard.
% Certains droits réservés : http://creativecommons.org/licenses/by-sa/3.0/
%------------------------------------
%------------------------------------
\newcommand{\Title}{Curriculum Vit\ae}
\newcommand{\Author}{Laurent Perrinet}
\newcommand{\Address}{Chercheur en neurosciences computationnelles (DR2 CNRS)\\ Institut de Neurosciences de la Timone\\
UMR 7289, CNRS / Aix-Marseille Université\\ 27, Bd. Jean Moulin, 13385 Marseille Cedex 5, France}%
\newcommand{\Website}{https://laurentperrinet.github.io}
\newcommand{\Email}{Laurent.Perrinet@univ-amu.fr}
%------------------------------------
\documentclass[11pt, a4paper, french]{article}
\usepackage{fontspec}
% MISE EN PAGE DU DOCUMENT
\usepackage{geometry}
\geometry{a4paper, textwidth=13cm, textheight=25cm, marginparsep=7pt, marginparwidth=3cm}
\setlength\parindent{0in}
% POLICES
\usepackage[usenames,dvipsnames]{color}
%\usepackage{xunicode}
\usepackage{xltxtra}
\defaultfontfeatures{Mapping=tex-text} % convertit les caractères spéciaux LaTeX (``guillemets'' --- tirets, etc.) en unicode
%\setromanfont [Ligatures={Common}, Numbers={OldStyle}]{Hoefler Text}
%\setmonofont[Scale=0.8]{Calibri}
%\setsansfont[Scale=0.9]{Optima Regular}
% ---- COMMANDES PERSONNALISÉES
\chardef\&="E050
\newcommand{\html}[1]{\href{#1}{\scriptsize\textsc{[html]}}}
\newcommand{\pdf}[1]{\href{#1}{\scriptsize\textsc{[pdf]}}}
\newcommand{\doi}[1]{\href{#1}{\scriptsize\textsc{[doi]}}}
% ---- ESPERLUETTE PERSONNALISÉE
%\newcommand{\amper}{{\fontspec[Scale=.95]{Hoefler Text}\selectfont\itshape\&}}
\newcommand{\amper}{{\selectfont\itshape\&}}
% ---- ANNÉES EN MARGE
\usepackage{marginnote}
%\newcommand{\amper{}}{\chardef\amper="E0BD }
\newcommand{\years}[1]{\marginnote{\scriptsize #1}}
\renewcommand*{\raggedleftmarginnote}{}
\setlength{\marginparsep}{7pt}
\reversemarginpar
% EN-TÊTES
\usepackage{sectsty}
\usepackage[normalem]{ulem}
\sectionfont{\sffamily\mdseries\large\underline}
\subsectionfont{\rmfamily\mdseries\scshape\normalsize}
\subsubsectionfont{\rmfamily\bfseries\upshape\normalsize}
% CONFIGURATION PDF
% ---- REMPLIR ICI LE TITRE DU DOCUMENT ET L'AUTEUR
\usepackage[bookmarks, colorlinks, breaklinks]{hyperref}  %dvipdfm,
\hypersetup{linkcolor=blue,citecolor=blue,filecolor=black,urlcolor=blue,
pdftitle={\Title},%
pdfauthor={\Author < \Email > \Address - \Website},%
pdfsubject={\Title}%
}%
\expandafter\ifx\csname urlstyle\endcsname\relax
  \providecommand{\doi}[1]{doi: #1}\else
  \providecommand{\doi}{doi: \begingroup \urlstyle{rm}\Url}\fi
% DOCUMENT
\begin{document}
\pagestyle{empty}
\reversemarginpar
\textsf{\Large \Author}\\[1cm]
\Address \\[.2cm]
%Téléphone : \texttt{609-734-8000}\\
%Fax : \texttt{609-924-8399}\\[.2cm]
%Email : \href{mailto:\Email}{\Email}\\
\textsc{url} : \href{\Website}{\Website}\\
%\vfill
\section*{Intérêts de recherche}
Mes recherches examinent les bases théoriques et empiriques de l'adaptation neurale, en me concentrant sur la manière dont les propriétés structurelles et fonctionnelles co-évoluent pour traiter de manière optimale les régularités statistiques des environnements naturels.
%\vfill
%Né : 23 février 1973 --- Bordeaux, France\\
%Nationalité : Française
%%%\hrule
%\section*{Poste actuel}
%\noindent\years{depuis 2004}\emph{Chercheur (CR1)}, Institut de Neurosciences de la Timone (INT).
%%\hrule
\section*{Domaines de spécialisation}
Inférence spatio-temporelle dans les aires sensorielles de bas niveau. %\\
Apprentissage non supervisé dans les cartes topographiques. %\\
Processus prédictifs et perception active.
%%%\hrule
%\section*{Postes occupés}
%
%\noindent\years{2010--12}Chercheur invité, UCL (Londres, Royaume-Uni). Groupe de neurobiologie théorique de Karl Friston.\\
%\noindent\years{1999}Chercheur invité, USAFB (Rome, NY) / Université de San Diego en Californie.\\
%\years{1997}Chercheur invité, Jet Propulsion Laboratory (NASA), Pasadena, Californie. Département des Sciences Terrestres, Laboratoire d'Imagerie Radar\\
%\years{9/1995-6/96}Ingénieur chez Alcatel, Vienne (Autriche). Département des Systèmes de Traitement de la Voix.\\
%
%\hrule
\section*{Formation}

\noindent\years{2014}\textsc{Habilitation à Diriger des Recherches} Aix-Marseille Université

\noindent\years{1999-2003}\textsc{Doctorat} en neurosciences cognitives, ONERA/DTIM, Toulouse (France)

\years{1993 - 1998}\textsc{Diplôme d'ingénieur} {\sc Supaéro} (Toulouse, France).

% \years{10/1999-2/2003}\textsc{Master} en sciences cognitives, ONERA/DTIM, Toulouse (France)
%
%%\hrule
%\section*{Subventions, distinctions et récompenses}
%\noindent\years{1921}Prix Nobel de Physique, Fondation Nobel
\section*{Publications sélectionnées}% \amper{} conférences}

\noindent\years{2024} Antoine Grimaldi, Laurent U Perrinet. ``Learning heterogeneous delays in a layer of spiking neurons for fast motion detection.'' {\bf Biological Cybernetics}.

\noindent\years{2023}Hugo Ladret, Nelson Cortes, Lamyae Ikan, Frédéric Chavane, Christian Casanova, Laurent U Perrinet. ``Cortical recurrence supports resilience to sensory variance in the primary visual cortex.'' {\bf Nature Communications Biology}.

\noindent\years{2021}Victor Boutin, Angelo Franciosini, Franck Ruffier, Frédéric Chavane and Laurent U Perrinet. ``Sparse Deep Predictive Coding captures contour integration capabilities of the early visual system.'' {\bf PLoS Computational Biology}.

\noindent\years{2020}Chloé Pasturel, Anna Montagnini and Laurent Perrinet. ``Humans adapt their anticipatory eye movements to the volatility of visual motion properties.'' {\bf PLoS Computational Biology}.

% \noindent\years{2019}Sandrine Chemla, Alexandre Reynaud, Matteo diVolo, Yann Zerlaut, Laurent Perrinet, Alain Destexhe and Frédéric Chavane.  ``Suppressive waves disambiguate the representation of long-range apparent motion in awake monkey V1.'' {\bf Journal of Neuroscience}.

%\noindent\years{2017}Mina A Khoei, Guillaume S Masson and Laurent Perrinet. ``The flash-lag effect as a motion-based predictive shift.'' {\bf PLoS Computational Biology}.

%\noindent\years{2015}Jonathan Vacher, Andrew Isaac Meso, Laurent Perrinet and Gabriel Peyré. ``Biologically Inspired Dynamic Textures for Probing Motion Perception.'' {\bf Advances in Neural Information Processing Systems}.

\noindent\years{2012}Karl Friston, Rick A. Adams, Laurent Perrinet and Michael Breakspear,  ``Perceptions as Hypotheses: Saccades as Experiments'', {\bf Front in Psychology}.

%\noindent\years{2012}Claudio Simoncini, Laurent Perrinet, Anna Montagnini, Pascal Mamassian and Guillaume Masson, ``More is not always better: dissociation between perception and action'', {\bf Nature Neuroscience}. %  explained by adaptive gain control

%\noindent\years{2012}Paula S. Leon, Ivo Vanzetta, Guillaume S. Masson and Laurent Perrinet, ``Motion Clouds: Model-based stimulus synthesis of natural-like random textures for the study of motion perception'', {\bf Journal of Neurophysiology}.%, 107(11):3217--3226.
%
%\noindent\years{2012}Laurent~Perrinet and Guillaume~S. Masson, ``Motion-based prediction is sufficient to solve the aperture problem", {\bf Neural Computation}. \\ %
%\noindent\years{2011}Guillaume~S. Masson  and Laurent~Perrinet, ``The behavioral receptive field underlying motion integration for primate tracking eye movements", {\bf Neuroscience and biobehavioral reviews}. \\ %

\noindent\years{2010}Laurent~Perrinet, ``Role of homeostasis in learning sparse representations", {\bf Neural Computation}.%, 22\penalty0 (7). \\ %

%\noindent\years{2007}Frederic Barthelemy, Laurent~Perrinet, Eric Castet, and  Guillaume~S. Masson, ``Dynamics of distributed {1{D}} and {2{D}} motion representations for short-latency ocular following", {\bf Vision {R}esearch}.\\%, 48\penalty0 (4):\penalty0 501--22. \\%\newblock \doi{10.1016/j.visres.2007.10.020}.\\

%\noindent\years{2007}Laurent~Perrinet and Guillaume~S. Masson, ``Modeling spatial integration in the ocular following response using a  probabilistic framework", {\bf Journal of {P}hysiology ({P}aris)}.\\%

%\noindent\years{2007}Sylvain Fischer, Rafael Redondo, Laurent~Perrinet, and Gabriel  Crist{\'o}bal, ``Sparse approximation of images inspired from the functional architecture of the primary visual areas'' {\bf EURASIP Journal on Advances in Signal Processing},  (1):122.\\

%\noindent\years{2004}Laurent Perrinet, “Feature detection using spikes : the greedy approach", {\bf Journal of Physiology, Paris}.\\

\noindent\years{2004}Laurent Perrinet, Manuel Samuelides and Simon Thorpe, ``Coding static natural images using spiking event times : do neurons cooperate?",  {\bf IEEE Transactions on Neural Networks}.%, 15(5):1164--75.%

{\footnotesize \href{https://github.com/laurentperrinet/perrinet_curriculum-vitae.tex}{Dernière mise à jour : \today},  \href{https://github.com/laurentperrinet/perrinet_curriculum-vitae.tex/blob/master/perrinet_curriculum-vitae-full.pdf}{Version complète}.
%\end{center}
\end{document}
