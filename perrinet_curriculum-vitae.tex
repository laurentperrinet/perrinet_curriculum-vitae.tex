%!TEX TS-program = lualatex
%!TEX encoding = UTF-8 Unicode
%------------------------------------
% Dario Taraborelli
% Typesetting your academic CV in LaTeX
%
% URL: http://nitens.org/taraborelli/cvtex
% DISCLAIMER: This template is provided for free and without any guarantee
% that it will correctly compile on your system if you have a non-standard
% configuration.
% Some rights reserved: http://creativecommons.org/licenses/by-sa/3.0/
%------------------------------------
%------------------------------------
\newcommand{\Title}{Curriculum Vit\ae}
\newcommand{\Author}{Laurent U Perrinet}
\newcommand{\Address}{Researcher in Computational Neuroscience\\ Institut de Neurosciences de la Timone\\
UMR 7289, CNRS / Aix-Marseille Universit{\'e}\\ 27, Bd. Jean Moulin, 13385 Marseille Cedex 5, France}%
\newcommand{\Website}{https://laurentperrinet.github.io}
\newcommand{\Email}{Laurent.Perrinet@univ-amu.fr}
%------------------------------------
\documentclass[11pt, a4paper]{article}
\usepackage{fontspec}
% DOCUMENT LAYOUT
\usepackage{geometry}
\geometry{a4paper, textwidth=13cm, textheight=24cm, marginparsep=7pt, marginparwidth=3cm}
\setlength\parindent{0in}
% FONTS
\usepackage[usenames,dvipsnames]{color}
%\usepackage{xunicode}
\usepackage{xltxtra}
\defaultfontfeatures{Mapping=tex-text} % converts LaTeX specials (``quotes'' --- dashes etc.) to unicode
%\setromanfont [Ligatures={Common}, Numbers={OldStyle}]{Hoefler Text}
%\setmonofont[Scale=0.8]{Calibri}
%\setsansfont[Scale=0.9]{Optima Regular}
% ---- CUSTOM COMMANDS
\chardef\&="E050
\newcommand{\html}[1]{\href{#1}{\scriptsize\textsc{[html]}}}
\newcommand{\pdf}[1]{\href{#1}{\scriptsize\textsc{[pdf]}}}
\newcommand{\doi}[1]{\href{#1}{\scriptsize\textsc{[doi]}}}
% ---- CUSTOM AMPERSAND
%\newcommand{\amper}{{\fontspec[Scale=.95]{Hoefler Text}\selectfont\itshape\&}}
\newcommand{\amper}{{\selectfont\itshape\&}}
% ---- MARGIN YEARS
\usepackage{marginnote}
%\newcommand{\amper{}}{\chardef\amper="E0BD }
\newcommand{\years}[1]{\marginnote{\scriptsize #1}}
\renewcommand*{\raggedleftmarginnote}{}
\setlength{\marginparsep}{7pt}
\reversemarginpar

% HEADINGS
\usepackage{sectsty}
\usepackage[normalem]{ulem}
\sectionfont{\sffamily\mdseries\large\underline}
\subsectionfont{\rmfamily\mdseries\scshape\normalsize}
\subsubsectionfont{\rmfamily\bfseries\upshape\normalsize}

% PDF SETUP
% ---- FILL IN HERE THE DOC TITLE AND AUTHOR
\usepackage[bookmarks, colorlinks, breaklinks]{hyperref}  %dvipdfm,
\hypersetup{linkcolor=blue,citecolor=blue,filecolor=black,urlcolor=blue,
pdftitle={\Title},%
pdfauthor={\Author < \Email > \Address - \Website},%
pdfsubject={\Title}%
}%
\expandafter\ifx\csname urlstyle\endcsname\relax
  \providecommand{\doi}[1]{doi: #1}\else
  \providecommand{\doi}{doi: \begingroup \urlstyle{rm}\Url}\fi

% DOCUMENT
\begin{document}
\reversemarginpar
\textsf{\LARGE \Author}\\[1cm]
\Address \\[.2cm]
%Phone: \texttt{609-734-8000}\\
%Fax: \texttt{609-924-8399}\\[.2cm]
email: \href{mailto:\Email}{\Email}\\
\textsc{url}: \href{\Website}{\Website}\\
\vfill
\section*{Research interests}
I am interested in bridging the gap between the structure and the function of neural systems by showing how they optimally adapt to the statistics of natural environments.
\vfill
Born: February 23rd, 1973 in Bordeaux, France\\
Nationality:  French

%%\hrule
\section*{Current position}
\noindent\years{since 2020}\emph{Researcher (DR2 CNRS)}, Institut de Neurosciences de la Timone (INT).\\
\noindent\years{2004-2020}\emph{Researcher (CR CNRS)}, Institut de Neurosciences de la Timone (INT).

%%\hrule
\section*{Areas of specialization}
Spatio-temporal inference in low-level sensory areas. \\
Unsupervised learning in topographic maps. \\
Predictive processes and active perception.

%%\hrule
\section*{Appointments held}

\noindent\years{2010--12}Visiting Scholar at Karl Friston theoretical neurobiology group, UCL (London, UK).\\
\noindent\years{2004}Research Scholar, with B. Olshausen / Redwood Neuroscience Center.\\
\noindent\years{1999}Research Scholar, USAFB (Rome, NY) / University of San Diego.\\
\years{1997}Research Scholar, Jet Propulsion Laboratory (Nasa), Pasadena, California. Department of Terrestrial Science, Imaging Radar Laboratory\\
\years{9/1995-6/96}Engineer at Alcatel, Vienna (Austria). Department of Voice Processing Systems.\\


%\hrule
\section*{Education}
\noindent\years{2014}\textsc{HDR} Aix-Marseille Université

\noindent\years{1999-2003}\textsc{PhD} in Cognitive Neuroscience, ONERA/DTIM, Toulouse (France)

\years{1993 - 1998}\textsc{MSc} in Engineering  {\sc Supaéro} (Toulouse, France), one of the leading French Engineering Schools ("Grandes Ecoles"). Specialization in stochastic models for signal and image processing.
% \years{10/1999-2/2003}\textsc{Master} in Cognitive Sciences, ONERA/DTIM, Toulouse (France)
%
%%\hrule
%\section*{Grants, honors \amper{} awards}

%\noindent\years{1921}Nobel Prize in Physics, Nobel Foundation

\section*{Selected publications}% \amper{} talks}

\subsection*{Journal articles}

\noindent\years{2021}Victor Boutin, Angelo Franciosini, Franck Ruffier, Frédéric Chavane and Laurent U Perrinet. ``Sparse Deep Predictive Coding captures contour integration capabilities of the early visual system.'' {\bf PLoS Computational Biology}.

\noindent\years{2020}Chloé Pasturel, Anna Montagnini and Laurent U Perrinet. ``Humans adapt their anticipatory eye movements to the volatility of visual motion properties.'' {\bf PLoS Computational Biology}.

\noindent\years{2019}Sandrine Chemla, Alexandre Reynaud, Matteo diVolo, Yann Zerlaut, Laurent U Perrinet, Alain Destexhe and Frédéric Chavane.  ``Suppressive waves disambiguate the representation of long-range apparent motion in awake monkey V1.'' {\bf Journal of Neuroscience}.

\noindent\years{2017}Mina A Khoei, Guillaume S Masson and Laurent U Perrinet. ``The flash-lag effect as a motion-based predictive shift.'' {\bf PLoS Computational Biology}.

\noindent\years{2015}Jonathan Vacher, Andrew Isaac Meso, Laurent U Perrinet and Gabriel Peyré. ``Biologically Inspired Dynamic Textures for Probing Motion Perception.'' {\bf Advances in Neural Information Processing Systems}.

\noindent\years{2012}Karl Friston, Rick A. Adams, Laurent U Perrinet and Michael Breakspear,  ``Perceptions as Hypotheses: Saccades as Experiments'', {\bf Frontiers in Psychology}.

\noindent\years{2012}Claudio Simoncini, Laurent U Perrinet, Anna Montagnini, Pascal Mamassian and Guillaume Masson, ``More is not always better: dissociation between perception and action explained by adaptive gain control'', {\bf Nature Neuroscience}.

\noindent\years{2012}Paula S. Leon, Ivo Vanzetta, Guillaume S. Masson and Laurent U Perrinet, ``Motion Clouds: Model-based stimulus synthesis of natural-like random textures for the study of motion perception'', {\bf Journal of Neurophysiology}.%, 107(11):3217--3226.

%\noindent\years{2012}Laurent~U. Perrinet and Guillaume~S. Masson, ``Motion-based prediction is sufficient to solve the aperture problem", {\bf Neural Computation}. \\ %
%\noindent\years{2011}Guillaume~S. Masson  and Laurent~U. Perrinet, ``The behavioral receptive field underlying motion integration for primate tracking eye movements", {\bf Neuroscience and biobehavioral reviews}. \\ %

\noindent\years{2010}Laurent~U. Perrinet, ``Role of homeostasis in learning sparse representations", {\bf Neural Computation}.\\%, 22\penalty0 (7). \\ %
%\noindent\years{2007}Frederic Barthelemy, Laurent~U. Perrinet, Eric Castet, and  Guillaume~S. Masson, ``Dynamics of distributed {1{D}} and {2{D}} motion representations for short-latency ocular following", {\bf Vision {R}esearch}.\\%, 48\penalty0 (4):\penalty0 501--22. \\%\newblock \doi{10.1016/j.visres.2007.10.020}.\\
%\noindent\years{2007}Laurent~U. Perrinet and Guillaume~S. Masson, ``Modeling spatial integration in the ocular following response using a  probabilistic framework", {\bf Journal of {P}hysiology ({P}aris)}.\\%
%\noindent\years{2007}Sylvain Fischer, Rafael Redondo, Laurent~U. Perrinet, and Gabriel  Crist{\'o}bal, ``Sparse approximation of images inspired from the functional architecture of the primary visual areas'' {\bf EURASIP Journal on Advances in Signal Processing},  (1):122.\\
%\noindent\years{2004}Laurent U Perrinet, “Feature detection using spikes : the greedy approach", {\bf Journal of Physiology, Paris}.\\

\noindent\years{2004}Laurent U Perrinet, Manuel Samuelides and Simon Thorpe, ``Coding static natural images using spiking event times : do neurons cooperate?",  {\bf IEEE Transactions on Neural Networks}.\\%, 15(5):1164--75.%
%\end{itemize}
%\subsection*{Book chapter}

%\noindent\years{2007}Laurent U Perrinet (2007) “Dynamical neural networks: modeling low-level vision at short latencies".  In \emph{Topics in Dynamical Neural Networks: From Large Scale Neural Networks to Motor Control and Vision}: volume 142 of {\bf The European Physical Journal (Special Topics)}.  Springer Berlin / Heidelberg.%, mar 2007, pages 163--225.  \doi{10.1140/epjst/e2007-00061-7}. %
\subsection*{Book}

\noindent\years{2015}Gabriel Cristobal, Laurent U Perrinet and Matthias S Keil, editors. ``Biologically Inspired Computer Vision.'' {\bf Wiley-VCH} % Verlag GmbH et Co. KGaA, 7 oct. 2015. isbn : 9783527680863.
doi : 10.1002/9783527680863. %url : http://onlinelibrary. wiley.com/book/10.1002/9783527680863
%
%\subsection*{Talks}
%
%\noindent\years{2008}Laurent U Perrinet (2008), “Adaptive sparse spike coding : applications of neuroscience to the compression of natural images." In \emph{Optical and Digital Image Processing Conference 7000 -  Proceedings of SPIE Volume 7000, April 2008}, pages 15 -- S4\\

%\vspace{1cm}
\vfill{}
\hrulefill

% FILL IN THE FULL URL TO YOUR CV
\begin{center}
%{\scriptsize  Last updated: \today\- •\-
%% ---- PLEASE LEAVE THIS BACKLINK FOR ATTRIBUTION AS PER CC-LICENSE
%Typeset in \href{http://nitens.org/taraborelli/cvtex}{
%% ---- FILL IN THE FULL URL TO YOUR CV HERE
%\href{http://nitens.org/taraborelli/cvtex}{http://nitens.org/taraborelli/cvtex}}

{\footnotesize \href{https://github.com/laurentperrinet/perrinet_curriculum-vitae.tex}{Last updated: \today .}
}
\end{center}


\end{document}
