%!TEX root = perrinet19cnrs.tex
%!TeX TS-program = Lualatex
%!TeX encoding = UTF-8 Unicode
%!TeX spellcheck = fr-FR
%!BIB TS-program = bibtex
% -*- coding: UTF-8; -*-
% vim: set fenc=utf-8
\chapter{Curriculum Vit\ae\ détaillé}
\section{Curriculum Vit\ae\ }
\subsection{Présentation rapide}%
%\subsection{ Curriculum Vit\ae\ : Laurent U. Perrinet}%
%\begin{resume}
\indent 46 ans, né le : 23-02-1973 à  Talence (Gironde).  %\\

\begin{itemize}
\item Affiliation: Chargé de Recherche (CRCN, CNRS), \Team\ - \Institute\ (\InstituteUMR)
\item Adresse: \Address 
\item E-mail: \url{mailto:\Email} 
\item Téléphone: 04 91 32 40 44
\item URL: \url{\Website}
\end{itemize}

%\subsection*{Bio-sketch}
%Chargé de Recherche (CR1, CNRS) since 2000. His scientific interests focus on bridging computational understanding of neural dynamics and low-level sensory processing by focusing on motion perception. He is the author of papers in machine learning, computational neuroscience and behavioral psychology. One key concept is the use of statistical regularities from natural scenes as a main drive to integrate local neural information into a global understanding of the scene. In a recent paper that he coauthored (in Nature Neuroscience) he develops a method to use synthesized stimuli targeted to analyze physiological data in a system-identification approach. He published 27 articles in international peer-reviewed journals.

%\vspace{.25in}
\subsection*{Objectifs de Recherche}

Mon objectif de recherche est d'étudier l'hypothèse selon laquelle on peut comprendre les liens entre la structure neurale et la fonction des systèmes sensoriels comme l'optimisation de leur adaptation aux statistiques des scènes naturelles par des processus de type prédicitf.

Plus précisément, je vise à étendre la compréhension des facultés sensorielles et cognitives sous la forme de modèles de réseaux de neurones impulsionnels qui réalisent de façon efficace des algorithmes de  perception visuelle. En effet, les brèves impulsions du potentiel de membrane se propageant au fil des neurones sont une caractéristique universelle des systèmes nerveux et permettent de construire des modèles événementiels efficaces de traitement dynamique de l'information. Dans un but fonctionnel, je désire notamment implanter dans ces modèles des stratégies d'inférence grâce à des mécanismes d'apprentissage auto-organisés fixant les relations spatio-temporelles entre les neurones. Dans une approche applicative, nous envisageons la création de nouveaux types d'algorithmes basés sur ces recherches.

\subsection*{Mots clés}
Perception, vision, détection du mouvement. Calcul parallèle événementiel, émergence dans les systèmes complexes, codage neuronal. Inférence bayesienne, minimisation de l'énergie libre, statistiques des scènes naturelles.

\subsection{Diplômes \& titres universitaires}

%\begin{itemize}
\textbf{Habilitation à Diriger des Recherches}, AMU, Marseille\hfill \years{\textbf{2014}}\\
\vspace*{-.15in}
\begin{itemize}
\item[] \'Ecole Doctorale Sciences de la Vie et de la Santé, Aix-Marseille Université, France. 
\item[] Sous le titre ``Codage prédictif dans les transformations visuo-motrices'', j'ai défendu mon Habilitation à Diriger des Recherches le 14 avril 2017.
\item[] Le jury était constitué des Prof. Laurent Madelain (Université Lille III), Dr. Alain Destexhe (Université Paris XI, Rapporteur), Prof. Gustavo Deco (Universitat Pompeu Fabra, Barcelona, Rapporteur), Dr. Guillaume Masson (Aix-Marseille Université), Dr. Viktor Jirsa (Aix-Marseille Université, Rapporteur) et du Prof. Jean-Louis Mege (Aix-Marseille Université).
\end{itemize} %


	 
%%%%%%%%%%%%%%%%%%%%%%%%%%%% These %%%%%%%%%%%%%%%%%%%%%%%%%%%%%%%%%%%%%%%%%%%
\vspace*{.3cm} 
	%\item[] 
	\textbf{Doctorat de Sciences cognitives} ONERA/DTIM, Toulouse \hfill \years{\textbf{1999-2003}} \\
\vspace*{-.15in}
\begin{itemize}
\item[] Titre : \emph{Comment déchiffrer le code impulsionnel de la Vision? \'Etude du flux parallèle, asynchrone et épars  dans le traitement visuel ultra-rapide.}  Allocataire d'une bourse MENRT, accueil à  l'ONERA/DTIM. 
\begin{itemize}
%\vspace*{.05in}
		\item % codirection
		Cette thèse a été initiée par les résultats de la collaboration pendant le stage de DEA. Elle a été dirigée par Manuel Samuelides (professeur à  \textsc{Supaéro} et chargé de recherche à  l'ONERA/DTIM) et co-dirigée par Simon Thorpe (directeur de recherche au \textsc{CerCo})
		\item Participation et présentations à  de nombreux colloques internationaux (IJCNN99, NeuroColt00, CNS00, CNS01, LFTNC01, ESANN02, NSI02). Participation aux écoles d'été ``EU Advanced Course in Computational Neuroscience'' à  Trieste (Italie) et ``Telluride Neuromorphic Workshop'' au Colorado (\'Etats-Unis).
		\item % + org. dynn / enseignement (supaero trex pir / ensica/ cea) / chap. dynn
		En parallèle, j'ai participé à  l'organisation d'une conférence sur les réseaux de neurones dynamiques (\textsc{Dynn}*2000). Je me suis aussi impliqué dans des activités d'enseignement (initiation à  la programmation sous Matlab et théorie de la probabilité) pour des élèves de première et deuxième année d'école d'ingénieur (à  \textsc{Supaéro} et à  l'\textsc{Ensica}, Toulouse) et des travaux dirigés de robotique (Traitement de l'image et reconnaissance d'objets au laboratoire d'Informatique et d'Automatique de \textsc{Supaéro}). %Enfin, je me suis impliqué dans la rédaction d'un chapitre d'un livre  
		\item % soutenance
		La thèse de doctorat a été soutenue le 7 février 2003 à  l'université Paul Sabatier avec la mention "Très honorable avec les félicitations du jury". Le jury était présidé par Michel Imbert (Prof. Université P. Sabatier) et constitué par Yves Burnod (Directeur de recherche à  l'\textsc{INSERM} U483) et Jeanny Hérault (Professeur à  l'INPG, Grenoble). 
	\end{itemize} %
\end{itemize} %


%%%%%%%%%%%%%%%%%%%%%%%%%%%% dea %%%%%%%%%%%%%%%%%%%%%%%%%%%%%%%%%%%%%%%%%%%
\vspace*{.3cm} 
	%\item[] 
	\textbf{DEA de Sciences cognitives} \hfill \years{\textbf{1998-1999}} \\ Univ. Paris VII, P. Sabatier, EHESS, Polytechnique, mention TB. Allocataire d'une bourse de DEA. 
	%%  (stage fin étude simon / dea paris - petitot / stage cert / 2?/26, mention bien)
	\begin{itemize}	
			\item  Assistant de recherche, ONERA/DTIM \years{3/1999-7/1999}(Département de Traitement de l'Image et de Modélisation), Toulouse (stage de DEA).  
			\begin{itemize}
				\item  \'Etude de l'apprentissage de type Hebbien de réseaux de neurones basés sur un codage par rang. 
				\item  Application à  la reconnaissance de textures visuelles.
			\end{itemize} %
			\item Assistant de recherche, USAFB (Rome, NY)  \years{7/1999-8/1999}/ University of San Diego in California (\'Etats-Unis). % https://www.rollingstone.com/music/music-news/19-worst-things-about-woodstock-99-176052/
			%\begin{itemize}\item 
			\'Etude de l'apprentissage autonome dans un système complexe de type automate cellulaire. %\item Application à  la modélisation du comportement d'un pilote d'avion.\end{itemize} %
	\end{itemize} %
	 
%%%%%%%%%%%%%%%%%%%%%%%%%%%% supaero %%%%%%%%%%%%%%%%%%%%%%%%%%%%%%%%%%%%%%%%%%%
\vspace*{.3cm} 
	%\item[] 
	\textbf{Diplôme d'ingénieur {\sc Supaéro}}, Toulouse, France.  \hfill \years{\textbf{1993-1998}}\\
Spécialisation dans le traitement du signal et de l'image et en particulier dans les techniques des réseaux de neurones artificiels. \\
\vspace*{-.1in}
	\begin{itemize}
	 	\item Projets individuels sur la perception visuelle, la reconnaissance de locuteur, la compression de la parole et sur la reconnaissance de caractères.
		\item Ingénieur \textsc{Alcatel}, Vienne (Autriche). \years{9/1995-6/1996} Département du \emph{Voice Processing Systems}. Ce 'stage long' volontaire, intégré à  une formation de \textsc{Supaéro} sur les systèmes industriels, impliquait toutes les étapes de la conception d'un système de messagerie téléphonique de technologie élevée : conception, prototype, contrôle de qualité et étude marketing. %Le stage s'intégrait.
		\item  Assistant de recherche, \textsc{Jet Propulsion Laboratory}  \years{4/1997-9/1997}(\textsc{Nasa}), Pasadena, Californie. Département des Sciences de la Terre, Laboratoire d'imagerie radar, \textsc{Interférométrie radar SAR appliquée à la géophysique}
		\begin{itemize}
			\item Programmation d'un processus de traitement d'images radar interférométriques SAR comprenant des corrections géographiques, une série de filtrages et un traitement d'interférométrie. 
			\item \'Etude et programmation d'un générateur de carte topographique.% (DEM).
			\item Traitement des images obtenues pour surveiller la déformation de la croûte terrestre. \'Etude des tremblements de terre de Landers (Californie) et de Gulan (Chine).
		\end{itemize} 
		\item Assistant de recherche, \textsc{CerCo}  \years{4/1998-9/1998}(CNRS, UMR5549), Toulouse (stage de fin d'études d'ingénieur). Développement d'un réseau de neurones asynchrone appliqué à  la reconnaissance de caractères. 
		\begin{itemize}
			\item Programmation du code du réseau de neurones asynchrones. 
			\item \'Etude et utilisation des statistiques non-paramétriques %\citep{Barbe95} 
pour l'utili\-sation d'un code basé sur le rang d'activation des neurones. \item Implantation d'une nouvelle règle d'apprentissage du réseau de neurones.
		\end{itemize} %
	\end{itemize}
	
\subsection{Expérience scientifique professionnelle}
%%%%%%%%%% CNRS %%%%%%%%%%%%%%%%%%%%%%%%%%
\vspace*{.1cm}

	\textbf{Chargé de Recherche Classe Normale}, INT/CNRS, Marseille  \years{\textbf{2019-\ldots}} \\
%\vspace*{-.7cm}
%\begin{list1}
%\item[] 
%Institut de Neurosciences de la Timone (INT).
	 % (\emph{sous réserve de financement}) 
	 Au 1er janvier 2019, j'ai intégré l'équipe \href{http://www.int.univ-amu.fr/spip.php?page=equipe&equipe=NeOpTo&lang=en}{NeOpTo} de Frédéric Chavane (DR, CNRS). J'implémente les modèles prédictifs dans des architectures bio-mimétiques.  


	\textbf{Chargé de Recherche grade 1}, INT/CNRS, Marseille  \years{\textbf{2012-2019}} \\
%\vspace*{-.7cm}
%\begin{list1}
%\item[] 
%Institute de Neurosciences Cognitives de la Méditerranée (INCM).
	 % (\emph{sous réserve de financement}) 
	 Au 1er janvier 2012, notre équipe a intégré l'\Institute\ (\InstituteUMR\ ) à  Marseille (direction Guillaume Masson). J'approfondis les modèles en me concentrant sur des modèles de codage probabiliste distribués appliqué à la boucle sensori-motrice.  


	%\item[] 
\textbf{Mission longue} \years{10/2010-01/2012}  Karl Friston's theoretical neurobiology group (The Wellcome Trust Centre for Neuroimaging, University College London, London, UK). Collaboration avec Karl Friston sur l'application de modèles d'énergie libre aux mouvements oculaires. 

	\textbf{Chargé de Recherche grade 2}, INCM/CNRS, Marseille  \years{\textbf{2004-2012}} \\
%\vspace*{-.7cm}
%\begin{list1}
%\item[] 
%Institute de Neurosciences Cognitives de la Méditerranée (INCM).
	 % (\emph{sous réserve de financement}) 
	 Sous la conduite de Guillaume Masson à  l'INCM à  Marseille, j'ai étudié des modèles spatio-temporels d'inférence dans des scènes naturelles en application de la compréhension des mouvements oculaires. 
%%%%%%%%%%%%%%%%%%%%%%%%%%%% post-doc %%%%%%%%%%%%%%%%%%%%%%%%%%%%%%%%%%%%%%%%%%%

%\vspace*{.3cm}
	%\item[] 
	\textbf{Post-doctorat}, Redwood Neuroscience Institute (RNI), \'Etats-Unis  \years{\textbf{2004}} \\	% San Francisco (
%\vspace*{-.7cm}
%\begin{list1}
%\item[] 
%Institute de Neurosciences Cognitives de la Méditerranée (INCM).
	 % (\emph{sous réserve de financement}) 
	 Sous la conduite de Bruno Olshausen, j'ai comparé des modèles standards d'apprentissage avec une méthode originale centrée sur les potentiels d'action. Notamment, j'ai développé une méthode générique évaluant l'importance des processus homéostatiques dans l'apprentissage non-supervisé, qui a conduit à une publication dans le journal Neural Computation (référence A20-\citep{Perrinet10shl}).   J'ai ensuite étendu ce modèle à  l'apprentissage spatio-temporels dans des flux video.
	 
\section{Enseignement, formation et diffusion de la culture scientifique} %{\small(journaux à  comité de lecture)}}

\subsection{Encadrement de thèse et post-doctorants} %

Dans la période de référence, j'ai eu l'occasion d'encadrer trois doctorants %et un post doctorant 
en tant qu'encadrant principal: %. À noter qu'\href{https://laurentperrinet.github.io/authors/hugo-ladret/}{Hugo Ladret} vient de commencer sa thèse sur un contrat dont je suis investigateur principal.
\begin{itemize}
%	\item Mina Khoei 	``Emerging properties in a neural field model implementing probabilistic prediction'' (PhD, 2011-2014)
%	\item Wahiba Taouali 	``Motion Integration By V1 Population'' (Post-Doc, 2012-03 / 2015-01) 
	\item \href{https://laurentperrinet.github.io/authors/victor-boutin/}{Victor Boutin} 	``Controlling an aerial robot by human semaphore gestures using a bio-inspired neural network'' (12/2016-12/2019)
	\item \href{https://laurentperrinet.github.io/authors/angelo-franciosini/}{Angelo Franciosini} ``Trajectories in natural images and the sensory processing of contours'' (2017 / 2021)
	\item \href{https://laurentperrinet.github.io/authors/hugo-ladret/}{Hugo Ladret} ``A multiscale cortical model to account for orientation selectivity in natural-like stimulations''  (co-direction avec Christian Casanova, en cotutelle avec l'Université de Montréal, 2019 / 2022)
\end{itemize}


Dans la même période de référence, j'ai l'occasion d'encadrer des étudiants en co-direction de thèse, notamment sur l'ANR \href{https://laurentperrinet.github.io/project/anr-rem/}{REM} et le contrat \href{https://laurentperrinet.github.io/project/pace-itn/}{PACE-ITN}: % FACETS-ITN:
\begin{itemize}
%	\item David Arbib 	OBV1: Sélectivité à l'orientation dans le cortex visuel primaire. (master student, 2016-01 / 2016-06)
	\item Jean-Bernard Damasse 	``Gaze orientation and learning'' (2014-2017)
%	\item Jean Spezia 	developing MotionClouds (undergrad, 2014-09 / 2014-12)
%	\item Jens Kremkow 	Correlating Excitation and Inhibition in Visual Cortical Circuits: Functional Consequences and Biological Feasibility (PhD, 2006-01 / 2009-05)
	\item Kiana Mansour Pour 	``Predicting and selecting sensory events: inference for smooth eye movements'' (2015 - 2018)
%	\item Nicole Voges 	Complex dynamics in recurrent cortical networks based on spatially realistic connectivities (PostDoc, 2008 / 2010)
%	\item Victor Boutin 	Controlling an aerial robot by human semaphore gestures using a bio-inspired neural network (PhD, 12/2016-12/2019)
%	\item Wahiba Taouali 	Motion Integration By V1 Population (Post-Doc, 2012-03 / 2015-01) 
\end{itemize}

\subsection{Participation à des activités grand public} %

\begin{itemize}

	\item Participation à des activités de dissémination aux des Journées de Neurologie de Langue Française (JNLF) : conférence invitée \years{2019} ``Des illusions aux hallucinations visuelles: une porte sur la perception" \url{https://laurentperrinet.github.io/talk/2019-04-18-jnlf/}. 

	\item Ecriture d'un article de dissémination dans \href{https://theconversation.com/illusions-et-hallucinations-visuelles-une-porte-sur-la-perception-117389}{``The conversation''} (6400 lectures au 3 septembre 2019). 

	\item Participation à des activités grand public: \href{https://laurentperrinet.github.io/talk/2019-01-10-polly-maggoo/}{Rencontre avec les collégiens marseillais}, \href{https://laurentperrinet.github.io/talk/2018-10-10-polly-maggoo/}{fête de la science}, participation à un jury autour de \href{https://laurentperrinet.github.io/talk/2017-11-17-festival-interferences/}{la société, la science et le cinéma}. 

%projet "ELASTICITE" présentation au 104 (Paris) le 5 décembre 2015
%Conseil scientifique
%Exposition
%projet "TRAMES" présentation à la Fondation Vasarely (Aix) en 2016
%Conseil scientifique
%
%* Projet artistique en collaboration avec Etienne Rey (artiste plasticien - friche Belle de Mai) dans le cadre de Marseille Provence capitale européenne de la culture 2013. Exposition à la fondation Vasarely (Aix-en-Provence) * Participation au réseau NeuroComp.fr
%
%projet "TROPIQUE", label "Marseille-Provence capitale européenne de la culture 2013" Conseil scientifique
% http://invibe.net/LaurentPerrinet/EtienneRey

	
	%
%	\item Participation à des activités grand public: conférence invitée\years{2016} ``Les illusions visuelles, un révélateur du fonctionnement de notre cerveau" \url{http://invibe.net/LaurentPerrinet/Presentations/2016-04-25_PollyMaggoo}. 
%
%	\item Séminaires d'ouverture aux neurosciences pour des mathématiciens \years{2016} (école du CIRM EDP et probas)  \url{http://invibe.net/LaurentPerrinet/Presentations/2016-07-07_EDP-proba}. 
%
%	\item Rencontres \years{2015-2016} en milieu scolaire avec l'association Polly Maggoo  (voir par exemple \url{http://invibe.net/LaurentPerrinet/Presentations/2016-04-25_PollyMaggoo}). 
%
%	\item Rencontres Internationales Sciences \& Cinémas\years{2015-2016} intervention en tant que scientifique (voir \url{http://invibe.net/LaurentPerrinet/Presentations/2016-11-20_PollyMaggoo }). 
	
\end{itemize}



\subsection{Collaboration artistique} %
%\item  Projet artistique en collaboration avec Etienne Rey (artiste plasticien à la friche Belle de Mai) :

En parallèle avec les actions grand public, je développe une collaboration active avec un artiste plasticien, Etienne Rey (friche Belle de Mai, Marseille, voir~\url{https://laurentperrinet.github.io/authors/etienne-rey/}). Sur la période de référence, nous avons produit plusieurs actions, entre autres:

\begin{itemize}
	\item Sans gravité – une poétique de l’air – Ardenome à Avignon (2019, \url{https://laurentperrinet.github.io/post/2019-06-22_ardemone/}), 		\item Instabilité (series) @ Art-O-Rama, Installation avec sérigraphie, dessin mural, lumière (2018, \url{https://laurentperrinet.github.io/post/2018-09-09_artorama/}), 
	\item \href{https://laurentperrinet.github.io/talk/2018-01-25-meetup-neuronautes/}{Meetup Art et Neurosciences} : conférence avec des étudiants de neurosciences . 

%	\item projet "TRAMES" présentation à la Fondation Vasarely (Aix)\years{2016},
%	\item projet "ELASTICITE" présentation au 104 (Paris)\years{2015},
%	\item projet "TROPIQUE", label "Marseille-Provence capitale européenne de la culture 2013" Conseil scientifique : Collaboration artistique avec\years{2011-2013} le plasticien \'Etienne Rey en préparation de Marseille MPM capitale de la culture européenne 2013. Résidence à l'IMERA (Marseille), présentation aux festivals d'Enghien-les-bains et Ososphère (Strasbourg). Organisation de l'installation \years{Juin 2013}de l'\oe uvre sur le site de l'INT. Exposition de l'installation à la fondation\years{Octobre 2013} Vasarely (Aix-en-Provence). 
\end{itemize}


\subsection{Enseignement} %

Cours magistraux de Neurosciences Computationnelles \years{2018} dans le cadre du programme de thèse Marseillais en Neurosciences \url{https://laurentperrinet.github.io/post/2018-03-26-cours-neuro-comp-fep/}.

\begin{itemize}
	\item An introduction to the field of Computational Neuroscience , 
	\item Probabilities, the Free-energy principle and Active Inference.
\end{itemize}


J'ai récemment pris part à une école d'été organisée par le centre de neurosciences computationnelles de Valparaiso au Chili. Les thèmes abordés au cours de cette école étaient:
\begin{itemize}
	\item adaptation comportementale : \url{https://laurentperrinet.github.io/talk/2019-01-18-laconeu/}, 
	\item compensation des délais : \url{https://laurentperrinet.github.io/talk/2019-01-17-laconeu/}
	\item modélisation Bayesienne : \url{https://laurentperrinet.github.io/talk/2019-01-16-laconeu/}
	\item tutoriel modélisation de réseaux spikant : \url{https://laurentperrinet.github.io/talk/2019-01-14-laconeu/}
 
\end{itemize}


%Cours magistraux de Neurosciences Computationnelles \years{2015} dans le cadre du programme de thèse Marseillais en Neurosciences \url{http://invibe.net/LaurentPerrinet/Presentations/2015-12-08_cours-NeuroComp}. Ces cours seront renouvelés en 2017.
%
%\begin{itemize}
%	\item An introduction to the field of Computational Neuroscience , 
%	\item Decoding of feature selectivity in neural activity: Concrete applications in visual data 
%\end{itemize}
%
%
%J'ai aussi pris par récemment à une école d'été organisée par le centre de neurosciences computationnelles de Valparaiso au Chili. Les thèmes abordés au cours de cette école étaient:
%\begin{itemize}
%	\item compensation des délais \url{http://invibe.net/LaurentPerrinet/Presentations/2017-01-18_LACONEU}, 
%	\item codage des images \url{http://invibe.net/LaurentPerrinet/Presentations/2017-01-19_LACONEU}
%	\item modélisation Bayesienne \url{http://invibe.net/LaurentPerrinet/Presentations/2017-01-20_LACONEU}
% 
%\end{itemize}

\section{Transfert technologique, relations industrielles et valorisation} %
%Vous présenterez, pour les 5 ou 10 derniers semestres :
%vos participations à des contrats de recherche : contrats ANR, contrats européens (indiquer l?intitulé long, l?acronyme ou le sigle, la durée et les partenaires financeurs), contrats industriels, autres... (préciser votre rôle, les partenaires et les montants, le thème et le contenu des travaux, leur portée et leur impact) ;
%vos participations à des projets de créations d?entreprises ;
%vos participations à des travaux donnant lieu à des dépôts de brevets ou à des développements de logiciels (préciser votre rôle, décrire le contenu des travaux, les difficultés rencontrées et surmontées, l?impact des brevets ou logiciels) ;


\subsection{Contrats} %
Dans la période de référence, j'ai eu l'occasion de collaborer sur plusieurs contrats de niveau national (ANR) et international :% (cf~\url{http://invibe.net/LaurentPerrinet/TagGrants}). 

\begin{itemize}
\item soit à titre de collaborateur :

\begin{itemize}
 \item \href{https://laurentperrinet.github.io/project/anr-horizontal-v1/}{ANR Horizontal-V1} (2017--2021): ``Connectivité Horizontale et Prédiction de Cohérences dans l’Intégration de Contour'' avec Yves Fregnac,  
 \item \href{https://laurentperrinet.github.io/project/anr-causal/}{ANR CausaL} (2017--2021): ``Cognitive Architectures of  Causal  Learning'' avec Andrea Brovelli,  
 \item \href{https://laurentperrinet.github.io/project/anr-predicteye/}{ANR PredictEye} (2018--2022) : ``Mapping and predicting trajectories for eye movements'', avec Guillaume Masson
	\item \href{https://laurentperrinet.github.io/project/pace-itn/}{PACE-ITN}: ITN Marie Curie network (2015--2019) avec Anna Montagnini.
%	\item   ANR BalaV1: Balanced states in area V1 (2013--2016) \url{http://invibe.net/LaurentPerrinet/TagAnrBalaV1}
%	\item ANR \href{https://laurentperrinet.github.io/project/anr-rem/}{REM} 
%	\item  	ANR REM : Renforcement et mouvements oculaires (2013--2016) \url{http://invibe.net/LaurentPerrinet/TagAnrRem}
%	\item  	ANR SPEED: Traitement de la vitesse dans les scènes visuelles naturelles (2013--2016) \url{http://invibe.net/LaurentPerrinet/TagAnrSpeed}
%	\item  	ANR TRAJECTORY (2016--2019) \url{https://laurentperrinet.github.io/project/anr-trajectory/}
%	\item  	BrainScaleS: Brain-inspired multiscale computation in neuromorphic hybrid systems (2011-2014) \url{http://invibe.net/LaurentPerrinet/TagBrainScales}
%	\item  	CODDE: understanding brain and behaviour (2008--2012) \url{http://invibe.net/LaurentPerrinet/TagCodde}
%	\item  	FACETS-ITN: From Neuroscience to neuro-inspired computing (2010--2013) \url{http://invibe.net/LaurentPerrinet/TagFacetsItn}
%	\item  	FACETS: Fast Analog Computing with Emergent Transient States (2006--2010) \url{http://invibe.net/LaurentPerrinet/TagFacets}
\end{itemize}

\item soit à titre d'investigateur principal :
\begin{itemize}
	\item  \href{https://laurentperrinet.github.io/project/doc-2-amu/}{PhD DOC2AMU}: An Excellence Fellowship, H2020 (Excellence Scientifique) : Actions Marie Sklodowska-Curie (IF, ITN, RISE, COFUND) (2016--2019) 

	\item \href{https://laurentperrinet.github.io/project/phd-icn/}{PhD ICN} A grant from the Ph.D. program in Integrative and Clinical Neuroscience (PhD position, 2017 / 2021).
	\item Contrats Doctoraux d’Aix-Marseille Université 2019-2022 (PhD position, 2019--2022).
	\item  \href{https://laurentperrinet.github.io/project/spikeai/}{SpikeAI}: laureat du Défi Biomimétisme (2019) Algorithmes événementiels d’Intelligence Artificielle / Event-Based Artificial Inteligence (2019).
	\item  \href{https://laurentperrinet.github.io/project/aprovis3D/}{aprovis3D}: aprovis3D: Event-Based Artificial Inteligence (2019--2023, coordination du projet Jean Martinet).

\end{itemize}

\end{itemize}

\subsection{Développements de logiciels} %

Nous développons plusieurs lignes de recherche pour appliquer nos résultats à des problèmes concrets, sous forme de logiciels \emph{open source}:
\subsubsection{Mouvements des yeux et mouvement} %
\begin{itemize}

	\item \url{https://github.com/invibe/ANEMO} : traitement du signal pour l'analyse des mouvements des yeux
	\item \url{http://github.com/NeuralEnsemble/MotionClouds}: génération de textures pour la perception du mouvement~\citep{Sanz12,Vacher15nips,Vacher16}
	\item \url{https://github.com/laurentperrinet/LeCheapEyeTracker} -- \url{https://github.com/laurentperrinet/CatchTheEye} : Oculomètre minimal
utilisant l'apprentissage profond
\end{itemize}

\subsubsection{Biologically Inspired Computer Vision} %
\begin{itemize}
	\item \url{https://github.com/laurentperrinet/openRetina}: caméra événementielle minimale
%	\item  collaboration avec Gabriel Crist\'obal (CSIC, Madrid), par exemple pour la classification d'emphysèmes~\citep{Nava13} % avec "Rodrigo Nava, J. Victor Marcos, Boris Escalante-Ram\'\irez, Gabriel Crist\'obal, Laurent U. Perrinet, Ra\'ul S. J. Estépar. Advances in Texture Analysis for Emphysema Classification. 8259:214--221, 2013"
%	\item  collaboration avec I3S (Sophia-Antipolis) pour la classification d'images industrielles en suivant la méthodolgie développée dans~\citep{PerrinetBednar15},
	\item \url{http://github.com/bicv/SLIP}: techniques de traitement de l'image, utilisé notamment dans~\citep{Perrinet15bicv,Ravello16droplets,PerrinetBednar15,Perrinet15eusipco,Perrinet16EUVIP}
	\item \url{http://github.com/bicv/LogGabor}: représentations multi-échelles~\citep{Fischer07,Fischer07cv}
	\item \url{http://github.com/bicv/SparseEdges}: codage épars (parcimonieux) d'images naturelles~\citep{Perrinet15bicv}
\end{itemize}



\section{Encadrement, animation et management de la recherche}

%Vous présenterez, pour les 5 ou 10 derniers semestres :
%vos responsabilités dans l?animation de programmes ou projets français, européens ou internationaux (préciser le type de programme ou de projet, son ampleur et son impact, et décrire votre rôle) ;
%vos responsabilités et vos activités de direction d?équipe ou de laboratoire (préciser le nombre de personnes) ;
%vos autres responsabilités ou activités collectives au sein du CNRS ou plus largement (management de la recherche, fonctions ou missions d?intérêt général, participation à des conseils scientifiques et autres commissions, participation à des comités de lecture, etc.) ;
%autres.


Outre ces responsabilités scientifiques, je participe à l'animation scientifique sous d'autres formes. Tout d'abord pour l'évaluation de la recherche par les chercheurs en tant que membre d'un comité éditorial. Je développe aussi des collaborations internationales et en même temps dans la vie sociale de l'organisme:


\begin{itemize}

%	\item Collaboration dans le cadre de l'ANR \href{https://laurentperrinet.github.io/project/anr-horizontal-v1/}{HORIZONTAL V1} sur un modèle intégré d'étude du code neural dans la rétine de rongeurs:  mise en place de stimuli adaptés sous forme de textures synthétiques aléatoires, enregistrements in vitro (Organisme: Institute of Complex Systems of Valparaiso, Chili).
%

	\item Scientific reports (Nature group) Membre du comité éditorial

	\item  Relecteur dans de nombreuses revues et conférences, voir \url{https://publons.com/author/1206845/laurent-u-perrinet#profile}

	\item Membre élu CLAS GLM de Marseille-Joseph Aiguier/Timone, responsable de la petite enfance.

\end{itemize}



%%%%%%%%%%%%%%%%%%%%%%%%%%%%%%%%%%%%%%%%%%%%%%%%%%%%%%%%%%%%%
\newpage
